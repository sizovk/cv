\documentclass[12pt,a4paper,sans]{moderncv}        % possible options include font size ('10pt', '11pt' and '12pt'), paper size ('a4paper', 'letterpaper', 'a5paper', 'legalpaper', 'executivepaper' and 'landscape') and font family ('sans' and 'roman')
\usepackage{multicol}
\usepackage{amsmath}
\newcommand{\rom}[1]{%
  \textup{\uppercase\expandafter{\romannumeral#1}}%
}

\title{CV}

\renewcommand{\baselinestretch}{0.75} 
% moderncv themes
\moderncvstyle{classic}                             % style options are 'casual' (default), 'classic', 'oldstyle' and 'banking'
\moderncvcolor{red}                               % color options 'blue' (default), 'orange', 'green', 'red', 'purple', 'grey' and 'black'
%\renewcommand{\familydefault}{\sfdefault}         % to set the default font; use '\sfdefault' for the default sans serif font, '\rmdefault' for the default roman one, or any tex font name
%\nopagenumbers{}                                  % uncomment to suppress automatic page numbering for CVs longer than one page

% character encoding
\usepackage[utf8]{inputenc}                       % if you are not using xelatex ou lualatex, replace by the encoding you are using
%\usepackage{CJKutf8}                              % if you need to use CJK to typeset your resume in Chinese, Japanese or Korean

% adjust the page margins
\usepackage[scale=0.75]{geometry}
 \geometry{top=0.7cm}
 \geometry{bottom=0.3cm}
 \geometry{left=0.9cm}
 \geometry{right=1.1cm}
\setlength{\hintscolumnwidth}{3cm}                % if you want to change the width of the column with the dates
%\setlength{\makecvtitlenamewidth}{10cm}           % for the 'classic' style, if you want to force the width allocated to your name and avoid line breaks. be careful though, the length is normally calculated to avoid any overlap with your personal info; use this at your own typographical risks...
% personal data
\firstname{Kirill}
\familyname{Sizov}                               % optional, remove / comment the line if not wanted
\address{}{Moscow, Russia}{+7~(929)~746~6348} % optional, remove / comment the line if not wanted; the "postcode city" and and "country" arguments can be omitted or provided empty
              % optional, remove / comment the line if not wanted
%\phone[fixed]{+2~(345)~678~901}                    % optional, remove / comment the line if not wanted
%\phone[fax]{+3~(456)~789~012}                      % optional, remove / comment the line if not wanted
                           % optional, remove / comment the line if not wanted
%\homepage{www.johndoe.com}                         % optional, remove / comment the line if not wanted
\extrainfo{sizovki@yandex.ru\\
github.com/sizovk
}               % optional, remove / comment the line if not wanted
%\photo[64pt][0.4pt]{picture}                       % optional, remove / comment the line if not wanted; '64pt' is the height the picture must be resized to, 0.4pt is the thickness of the frame around it (put it to 0pt for no frame) and 'picture' is the name of the picture file
%\quote{Some quote}                                 % optional, remove / comment the line if not wanted

% to show numerical labels in the bibliography (default is to show no labels); only useful if you make citations in your resume
%\makeatletter
%\renewcommand*{\bibliographyitemlabel}{\@biblabel{\arabic{enumiv}}}
%\makeatother
%\renewcommand*{\bibliographyitemlabel}{[\arabic{enumiv}]}% CONSIDER REPLACING THE ABOVE BY THIS

% bibliography with mutiple entries
%\usepackage{multibib}
%\newcites{book,misc}{{Books},{Others}}
%----------------------------------------------------------------------------------
%            content
%----------------------------------------------------------------------------------
\begin{document}
\makecvtitle
\vspace{-4ex}
\section{Education}


\cventry{2019--2023}{BS in Applied Mathematics and Computer Science}{}{}{National Research University "Higher School of Economics", Department of Computer Science, Moscow, Russia}
{\underline{\normalsize Relevant courses:}}

\begin{cvcolumns}
\cvcolumn{}{
\begin{itemize}
    \item Algorithms and Data Structures
    \item C++ programming
    \item Calculus  
    \end{itemize}
}
\cvcolumn{}{
    \begin{itemize}
    \item Discrete mathematics
    \item Linear algebra
    \item Probability theory
    \end{itemize}
}
\end{cvcolumns}

% arguments 3 to 6 can be left empty
%\cventry{year--year}{Degree}{Institution}{City}{\textit{Grade}}{Description}


\section{Skills}
\cvitem{Programming languages}{C++, Python, SQL}
\cvitem{Applications}{Bash, Git, Subversion}
\cvitem{Languages}{Russian (Native), English (Intermediate)}
\cvitem{Other}{Linux experience, \LaTeX}


\section{Work experience}

\cvitem{Jul. 2019 -\\ Nov. 2019}{
    \textbf{Yandex, Software Engineer intern}
}


\begin{cvcolumns}
\cvcolumn{}{
    \begin{itemize}
        \item Python, SQL, MapReduce
        \item Made some MapReduce tasks on big advertising data.
        \item Developed self-monitoring system for checking time delays.
    \end{itemize}
}\end{cvcolumns}


\cvitem{Jul. 2020 -\\ Oct. 2020}{
    \textbf{Yandex, Software Engineer intern}
}

\begin{cvcolumns}
    \cvcolumn{}{
        \begin{itemize}
            \item C++, Python, Protobuf
            \item Made converter between protobuf and json format.
            \item Wrote protection from CRLF injection.
            \item Rewrote large code base from python2 to python3.
        \end{itemize}
}\end{cvcolumns}

\cvitem{Oct. 2020 -\\ Present time}{
    \textbf{Association of Olympic winners, Teacher}
}

\begin{cvcolumns}
    \cvcolumn{}{
        \begin{itemize}
            \item Teach to school students Python and some web programming.
        \end{itemize}
}\end{cvcolumns}

\section{Projects}

\cvitem{Summer 2020}{
    \textbf{Nutritive bot}
}

\begin{cvcolumns}
    \cvcolumn{}{
        \begin{itemize}
            \item Telegram bot for calculating individual daily intake of nutrients (vitamins, minerals, etc.).
            \item Bot link: \url{https://t.me/nutritive_bot}
            \item Source code: \url{https://github.com/sizovk/nutritive_bot}
        \end{itemize}
}\end{cvcolumns}

\cvitem{Summer 2020}{
    \textbf{Forms bot}
}

\begin{cvcolumns}
    \cvcolumn{}{
        \begin{itemize}
            \item Telegram bot for creating and completing forms.
            \item Bot link: \url{https://t.me/quick_forms_bot}
            \item Source code: \url{https://github.com/MakArtKar/forms_bot}
        \end{itemize}
}\end{cvcolumns}

\section{Achievements, awards and scholarships}
\cvitem{2018}{2nd degree award in \rom{19} All-Russian Team Olympiad in programming}
\cvitem{2018}{2nd degree award in \rom{7} Open School Olympiad in programming}
\cvitem{2017}{Award in the Russian National Olympiad in Mathematics}

\end{document}